\documentclass[final]{beamer}

\usetheme{local}
%\usetheme{Berlin}

\usepackage{fancyvrb}

\setbeamerfont{itemize}{size=\normalsize}
\setbeamerfont{itemize/enumerate body}{size=\normalsize}
\setbeamerfont{itemize/enumerate subbody}{size=\normalsize}
\setbeamersize{text margin left=1cm, text margin right=1cm}
% \beamertemplategridbackground[1cm]

\usepackage[latin1]{inputenc}
\usepackage[T1]{fontenc}
\usepackage[orientation=landscape,size=A0,grid,debug]{beamerposter}
\usepackage{array}

\usepackage{amsmath,amsthm, amssymb, latexsym}
\usepackage{exscale}
\usepackage{tikz}

% Programming Languages
\newcommand{\proglang}[1]{\textsf{#1}}
\newcommand{\RLang}{\proglang{R}}
\newcommand{\Stan}{\proglang{Stan}}
\newcommand{\pkg}[1]{\textbf{#1}}

% Math operators, misc
\DeclareMathOperator{\E}{E}
\DeclareMathOperator{\Median}{Median}
\DeclareMathOperator{\Var}{Var}
\DeclareMathOperator{\Cov}{Cov}
\DeclareMathOperator{\diag}{diag}
\newcommand{\tran}{^\top}

% Used to typeset distributions
\newcommand{\dist}[1]{\mathcal{#1}}
\newcommand{\paren}[1]{\ensuremath{\left(#1\right)}}
\newcommand{\dnorm}[1]{\ensuremath{\dist{N}\paren{#1}}}
\newcommand{\dmvnorm}[2]{\ensuremath{\dist{N}_{#2}\paren{#1}}}
\newcommand{\dt}[2]{\ensuremath{\dist{T}_{#1}\paren{#2}}}
\newcommand{\dcauchy}[1]{\ensuremath{\dist{C}\paren{#1}}}
\newcommand{\dhalfcauchy}[1]{\ensuremath{\dist{C}^{+}\paren{#1}}}
\newcommand{\dbeta}[1]{\ensuremath{\dist{B}\paren{#1}}}
\newcommand{\dinvbeta}[1]{\ensuremath{\dist{IB}\paren{#1}}}
\newcommand{\dgamma}[1]{\ensuremath{\dist{G}\paren{#1}}}
\newcommand{\dinvgamma}[1]{\ensuremath{\dist{IG}\paren{#1}}}
\newcommand{\dwishart}[1]{\ensuremath{\dist{W}\paren{#1}}}
\newcommand{\dinvwishart}[1]{\ensuremath{\dist{IW}\paren{#1}}}
\newcommand{\dunif}[1]{\ensuremath{\dist{U}\paren{#1}}}
\newcommand{\dexp}[1]{\ensuremath{\dist{E}\paren{#1}}}

\usepackage{arev}
\usepackage{eulervm}

\newcommand{\Model}[2]{$\mathcal{M}$(#1, #2)}

\def \ColOne {.15\textwidth}
\def \ColTwo {.32\textwidth}

 % Title, Author, Institute
\title{Bayesian Shrinkage in Dynamic Linear Models}
\subtitle{Unifying Smoothing and Structural Break Models}
\author{Jeffrey B. Arnold}
\newcommand{\authoremail}{jeffrey.arnold@gmail.com}
\institute[University of Rochester]{University of Rochester}
\date[2013-07-18]{July 18, 2013}

\begin{document}

\begin{frame}[fragile]
  \begin{columns}[t]

    \begin{column}{\ColTwo}

      \begin{block}{Sparse Time Varying Parameters}
        \begin{align*}
          y_{t} &\sim \mu_{t} + \nu_{t} & \nu \sim \dnorm{0, \sigma^{2}} \\
          \mu_{t} & \sim \mu_{t-1} + \omega_{t}
        \end{align*}
        where $\omega_{t}$ may be \textbf{sparse}, most $\omega_{t} \approx 0$ with a few $\omega_{t} \gg 0$.
      \end{block}

      \begin{block}{Distribution of the system variance ($\omega_{t}$)}
        To model sparse time-varying parameters, use a shrinkage prior for the distribution of $\omega_{t}$.
        Almost all Bayesian shrinkage priors can be represented as global-local scale-mixtures of normal distributions (Polson and Scott 2010),
        \begin{align*}
          \omega_{t} & \sim \dnorm{0, \sigma^{2} \tau^{2} \lambda^{2}_{t}} \\
        \end{align*}
        which differ in the prior distributions of $\tau^{2}$ and $\lambda^{2}_{t}$.
        This class includes the Laplacian and t-distributions.

        \vspace{1ex}
        This paper will use the \textbf{horseshoe prior} (Carvalho, Polson, and Lopes 2009),
        \begin{align*}
          \lambda_{t} &\sim \dhalfcauchy{0, 1}
        \end{align*}
        
        \vspace{1ex}
        The horseshoe prior distribution density compared to other distributions,
        \begin{center}
          \includegraphics{assets/fig-horseshoe1}
        \end{center}
      \end{block}

      \begin{block}{Shrinkage of $\theta_{t}$}
        Suppose $\theta_{t - 1} | y_{1:t - 1} \sim \dnorm{m_{t-1}, C_{t-1}}$, then
        \begin{equation}
          m_{t} = \E (p(\theta_{t} | y_{1:t})) = m_{t-1} + (1 - \delta_{t}) (y_{t} - m_{t-1})
        \end{equation}
        where $\delta_{t}$ is the shrinkage parameter,
        \begin{equation}
          \delta_{t} = \frac{V_{t}}{W_{t} + C_{t} + V_{t}}
        \end{equation}
        As $\delta_{t} \to 1$, $m_{t} \to m_{t - 1}$. As $W_{t} \to \infty$, $\delta_{t} \to 0$.

        \vspace{1ex}
        The distribution of $W_{t}$ implies a distribution of $\delta_{t}$.
        Suppose $W_{t}$ distributed horseshoe, $V_{t} = 1$, $C_{t} = 0$, and $\tau^{2} = 1$,
        then $\delta_{t} \sim \dbeta{\frac{1}{2}, \frac{1}{2}}$,

        \begin{center}
          \includegraphics{assets/fig-delta}          
        \end{center}

      \end{block}

      \begin{block}{Features}
        \begin{itemize}

        \item \textbf{Continuous model expansion} of smoothing and change point (structural break models)
        \item \textbf{Flexibile} extends to the more general Dynamic Linear Model case
          \begin{align*}
            y_{t} &= F_{t} \tran \theta_{t} + \nu_{t} & \nu_{t} & \sim \dnorm{0, V_{t}} \\
            \theta_{t} &= G_{t} \theta_{t - 1} + \omega_{t} & \omega_{t} & \sim \dnorm{0, W_{t}} \\
          \end{align*}
          This incorporates multivariate $y_{t}$, most time series models, and regression models.
        \item \textbf{Implementable} in standard software
        \item No assumption of a fixed number of structural breaks
        \item Allows for structural breaks in multiple parameters
        \item No assumption $\omega_{t} = 0$ in order to capture scale differences
        \item Focuses on the feature of substantive interest, the distribution of $\theta_{t} - \theta_{t-1}$, not $t$.
        \end{itemize}
      \end{block}

    \end{column}

    \begin{column}{\ColTwo}
      \begin{block}{Nile River Flow}
        \begin{columns}
          \begin{column}{0.5\textwidth}
            \includegraphics{assets/fig-nile}            
          \end{column}
          \begin{column}{0.5\textwidth}
            Annual flow of the Nile river at Aswan. The Aswan high dam
            was built in 1898.  (Cobb 1978, Durbin and Koopmans 2001)

            \vspace{1ex}
            For all models,
            \begin{align*}
              y_{t} &\sim \dnorm{\theta_{t}, V} \\
              \theta_{t} & \sim \dnorm{\theta_{t-1}, V W_{t}}
            \end{align*}
            where the value of $W_{t}$ differs between models,
            \vspace{1ex}
            \begin{center}
            \begin{tabular}{p{2.5in}p{4.5in}}
              model & $W_t$ \\
              \hline{}%
              \texttt{normal\_1} & $\tau^{2}$ \\
              \texttt{normal\_2} & $10^{6}$ for $t$ of year 1899, else $W_{t} = \tau^{2}$ \\
              \texttt{hs} & $\tau^{2} \lambda^{2}$, $\lambda_{t} \sim \dhalfcauchy{0, 1}$.
            \end{tabular}

            \begin{block}{Model fits}
              % latex table generated in R 3.0.0 by xtable 1.7-1 package
% Tue Jun 18 12:06:59 2013
\begin{tabular}{rrrrrr}
  \hline
 & WAIC & $L$ & $p$ & RMSE & $\chi^2$ \\ 
  \hline
\Model{Nile}{HS} & 1266 & -631 & 1.5 & 127 & 96 \\ 
  \Model{Nile}{normal} & 1282 & -638 & 2.9 & 107 & 65 \\ 
  \Model{Nile}{inter} & 1270 & -634 & 1.6 & 122 & 88 \\ 
   \hline
\end{tabular}
              
            \end{block}

            \end{center}

          \end{column}
        \end{columns}

      \end{block}

      \begin{block}{UK Driving Deaths}
        \begin{columns}
          \begin{column}{0.5\textwidth}
            \includegraphics{assets/fig-uk-drivers}
          \end{column}
          \begin{column}{0.5\textwidth}

            The monthly totals of car drivers in Great Britain killed, January 1969--December 1984. A law requiring the wearing of seatbelts went into effect in January 1983.

            \vspace{1ex}
            All models have a stochastic level and deterministic monthly components.
            \begin{align*}
              y_{t} &\sim N(\mu_{t} + \gamma_{1,t}, V) \\
              \mu_{t} &\sim N(\mu_{t-1}, V W_{t}) \\
              \gamma_{1,t} & = - \sum_{j = 1}^{11} \gamma_{j, t - 1} \\
              \gamma_{j,t} & =  \gamma_{j - 1, t - 1} & \text{for $j = 2:11$}
            \end{align*}

            \vspace{1ex}
            The models differ in the distribution of $W_{t}$,
            \begin{center}
              \vspace{1ex}
              \begin{tabular}{p{2.5in}p{4.5in}}
                model & $W_t$ \\
                \hline{}%
                \texttt{seatbelts1} & $\tau^{2}$ \\
                \texttt{seatbelts2} & $\tau^{2} \lambda^{2}$, $\lambda_{t} \sim \dhalfcauchy{0, 1}$ \\
                \texttt{seatbelts3} & $10^{6}$ for $t$ of January 1983, else $W_{t} = \tau^{2}$
              \end{tabular}

              \begin{block}{Model fits}
                % latex table generated in R 3.0.0 by xtable 1.7-1 package
% Thu Jul 18 03:25:29 2013
\begin{tabular}{rrrrr}
  \hline
 & WAIC & $L$ & $p$ & RMSE \\ 
  \hline
\texttt{seatbelts1} & -157 & 81 & 2.7 &  0 \\ 
  \texttt{seatbelts2} & -193 & 109 & 12.7 &  0 \\ 
  \texttt{seatbelts3} & -162 & 83 & 2.5 &  0 \\ 
   \hline
\end{tabular}
                
              \end{block}

            \end{center}
          \end{column}
        \end{columns}
      \end{block}

    \end{column}

    \begin{column}{\ColTwo}
      \begin{block}{Estimating DLMs}
        \begin{quote}
          This class of model [Dynamic Linear Models] is usually best handled not
          in general purpose computer programs for MCMC-based Bayesian analysis such
          as WinBUGS or JAGS. ... My experience is that tens
          to hundreds of millions of iterations --- taking several days of CPU time on even
          fast desktop computers --- are required so as to satisfactorily explore the posterior
          density for problems with large T using \texttt{WinBUGS} or \texttt{JAGS}. 
          (Jackman 2009, p. 477)
        \end{quote}
      \end{block}
      
      \begin{block}{Two-Step Practical Algorithm}
        Let $\theta$ be the latent states, and $\phi$ be all other parameters.
        
        \begin{enumerate}
        \item Take a sample $\left\{ \phi \right\}_{1:n}$ from $p(\phi | y_{t})$ using NUTS Hamiltonian Monte Carlo (HMC) in \Stan{}.
          Calculate log-likelihood contribution of $p(y_{t} | \phi)$ for DLM using Kalman Filter.
        \item For each $i$ in $\left\{ \phi \right\}_{1:n}$, draw one sample $p(\theta | y, \phi)$ \textit{en bloc} using Forward Filter Backwards Sample (FFBS) algorithm.
        \end{enumerate}
      \end{block}

      \begin{block}{Sampling parameters $p(\phi | y)$}
          Log-likelihood of DLM is calculated with the Kalman filter and written in in pure \Stan{} code.
          A library of reusable \Stan{} code is provided as CPP macros (\url{http://github.com/jrnold/stan_macros}).

          \vspace{1em}
          \BVerbatimInput[fontfamily=tt]{local_level_hs.stanpp}
          
      \end{block}

      \begin{block}{Sampling latent states $p(\theta | y, \phi)$}

        Uses new \RLang{} library \textbf{mcmcdb} (\url{http://github.com/jrnold/mcmcdb}) for easy post-processing of MCMC samples.

        \vspace{1em}
        \BVerbatimInput[fontfamily=tt]{example.R}
        
      \end{block}

    \end{column}
  \end{columns}
\end{frame}
\end{document}
