\documentclass{article}

\usepackage{amsmath}
\usepackage{amsfonts}
\usepackage[margin=1in]{geometry}

\usepackage{setspace}
\doublespace

\author{Jeffrey B. Arnold}
\title{Time Varying Parameter Estimation Robust to Structural Breaks}
\date{March 4, 2013}

\begin{document}

\maketitle{}

When modeling time-varying parameters, researchers have to choose between approaches which model the change in the parameter as a smooth, continuous process, such as dynamic linear models, or those which model the changes in the parameter as discrete jumps, i.e. structural break, change-point, or regime-shift models (\textit{ed. what is the best name to use?})
Change-point models are appealing because the change-points are often easy to substantively interpret, as they can be tied to events.
Change-point models are problematic because they either require that the researcher specify the number of change-points, and methods to estimate the number of change-points are not straight-forward (\textit{ed. I need a better reason than this}).
Another issue with change-point models is how they perform when the model is wrong; i.e. what if there are no change-points but a time-varying parameter.

This paper proposes a novel approach to modeling change-points within the dynamic linear model framework.
The problem of modeling the change in a parameter value is similar to that of estimating sparse parameters, when the number of parameters is large relative to the number of data points.
That problem is relatively well studied, and there is a growing literature on Bayesian shrinkage priors.
This paper adopts one of this distributions, the Horseshoe Prior (Carvallho) et al. 

\section{Time Varying Parameter Models}
\label{sec:time-vary-param}

A dynamic linear model (DLM) is defined by the following set of equations \parencites{WestHarrison1997}{CommandeurKoopman2007.pdf},
\begin{equation}
  \label{eq:9}
  \begin{aligned}[t]
    y_{t} &= F \theta_{t} + \nu_{t} \\
    \theta_{t} &= G \theta_{t - 1} + \omega_{t} 
  \end{aligned}
\end{equation}
and an initial prior distribution for $\theta_{0}$.
This represents a general class of models that includes linear regression (with constant or time-varying parameters) as well as standard ARIMA time-series models.

For the simplicity, in this section I only consider a special case of the DLM with a single parameter with a time-varying level. 
However, an advantage of this method, and working with state-space models in general, is that these methods can be extend to multiple variables, and varying slopes, etc.
A local-level model is a simple model describing parameter change over time,
\begin{equation}
  \label{eq:8}
  \begin{aligned}
    y_t &= \theta_t + \nu_t \\
    \theta_t &= \theta_{t-1} + \omega_t
  \end{aligned}
\end{equation}
where $v_{t} \sim N(0, \sigma^{2})$ and $E(\omega_t) = 0$.  

From \eqref{eq:8}, it is clear that the change in $\theta$ is determined by the distribution of $\omega_{t}$.
In a constant model, $\theta_{t} = \bar{\theta}$ for all $t$. 
This occurs when is not time-varying and $\omega_{t} = \delta_{0}$, where $\delta_{0}$ is the degenerate distribution at zero.

The nature of the change in $\theta$, i.e. $\theta_{t} - \theta_{t-1}$ is determined by the distribution of $\omega_t$.  
In the normal (Gaussian) linear dynamic model, $\omega_{t}$ are distributed i.i.d. normal,
\begin{equation}
  \label{eq:2}
  \omega_{t} \sim N(0, \tau^{2})
\end{equation}
Since the normal distribution does not have thick tails, it smooths the evolution of $\theta$ over time.
The use of the normal distribution shrinks the values of $\omega_{t}$, making large jumps unlikely, 
and also means that $\omega \neq 0$ with certainty.

The problem of estimating innovations $\omega_{t}$ is similar to problems of variable selection and shrinkage.

A ``structural break'' or ``change-point'' formulation can be incorporated into a dynamic linear model if $\omega_{t}$ is modeled as a spike-and-slab mixture distribution,
\begin{equation}
  \label{eq:1}
  \omega_{t} \sim p g(\omega_{i}) + (1 - p) \delta_{0} \text{,}
\end{equation}
where $p$ is the prior probability that $\omega_{t} \neq 0$, i.e. there is a regime shift, and 
$g(\omega_{t})$ is a distribution, usually non-information, over the change of the change in $\omega_{t}$ if there is a structural break.



A structural break model is a model which imposes sparsity on $\omega_{t}$. 
In other words, for most $t$, $\omega_{t} = 0$, while for a small number of $t$, $\omega_{t} \neq 0$.

A alternative approach is to model $\omega$ with a scale mixture of normal distributions.
\begin{equation}
  \label{eq:6}
  \begin{aligned}[t]
    \omega_{t} | \tau^{2}, \lambda^{2} & \sim N(0, \tau^{2} \lambda_{t}^{2}) \\
    \lambda_{t}^{2} & \sim p(\lambda^{2}_{t})
  \end{aligned}
\end{equation}
where $\tau^{2}$ is a global shrinkage parameter, and $\lambda_{t}^{2}$ are global shrinkage parameters.
Many Bayesian shrinkage priors can be expressed in this form. 

The class of scale-normal mixtures includes many of the suggested Bayesian shrinkage priors.

The $t$-distribution has suggested for dynamic linear model estimation that is robust to structural breaks.

This paper will focus on one particular shrinkage prior, the Horseshoe Prior distribution.

The Horseshoe prior distribution has several characteristics which make it particularly appealing for this problem.

This distribution has several characteristics that are useful.

If the 

\begin{equation}
  \label{eq:10}
  E(\omega_{t} | y_{t}, \theta_{t-1}, \lambda_{t}, \tau, \sigma^{2}) =
  \left(
    1 - \frac{1}{1 + \lambda_{t}^{2} \tau^{2}}
  \right) (y_{t} - \theta_{t - 1})
\end{equation}
For large $\psi^{2}$, the posterior mean $\hat\omega_{t}$ under the mixture innovation model is 
approximately $p_{t} (y_{t} - \theta_{t-1})$, where $p_{t}$ is the posterior probability that $\omega_{t} \neq 0$.
In the horseshoe estimator the quantity $\hat \omega_{t} = (1 - \hat \kappa) (y_{i} - \theta_{t})$.
The shrinkage weight behaves similarly to the posterior structural break probability $p_{t}$.
Because of this similarity, Carvalho proposes a simple decision rule under a symmetric 0-1 loss function
to identify signals, which in this case will be structural breaks,
\begin{equation}
  \label{eq:11}
  \text{Reject $H_{0t}$ if $p_{t} = 1 - E(\kappa_{t} | y_{t}, \theta_{t-1}, \tau, \omega^{2})$} > \frac{1}{2}
\end{equation}




\section{Examples}
\label{sec:examples}

I consider the performance of 

\subsection{Nile Flow Data}
\label{sec:nile}

\subsection{Greenbacks}
\label{sec:greenbacks-graybacks}

\subsection{Election}
\label{sec:election}



\end{document}

%%% Local Variables: 
%%% mode: latex
%%% TeX-master: t
%%% End: 

%  LocalWords:  Carvallho
