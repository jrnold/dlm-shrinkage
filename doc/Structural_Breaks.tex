\documentclass{article}

\usepackage{amsmath}
\usepackage{amsfonts}
\usepackage[margin=1in]{geometry}

\usepackage[style=authoryear]{biblatex}

\bibliography{local}

\usepackage{setspace}
\doublespace

\author{Jeffrey B. Arnold}
\title{Time Varying Parameter Estimation Robust to Structural Breaks}

\begin{document}

\maketitle{}

% From Koop and Potter (2004)
% here are two main
% approaches: one can estimate a model with a small number of change-points
% (usually one or two). Alternatively, one can estimate a time varying parameter (TVP) model where the parameters are allowed to change with each
% new observation, usually according to a random walk. A TVP model can
% be interpreted as having T − 1 breaks in a sample of size T.

When modeling time-varying parameters, researchers have to choose between approaches which model the change in the parameter as a smooth, continuous process, such as dynamic linear models, or those which model the changes in the parameter as discrete jumps, i.e. structural break, change-point, or regime-shift models.
This paper seeks to unify these approaches by using ...

Change-point models are appealing because the change-points are often easy to substantively interpret, as they can be tied to events.
Change-point models are problematic because they either require that the researcher specify the number of change-points, and methods to estimate the number of change-points are not straight-forward.
Another issue with change-point models is how they perform when the model is wrong; i.e. what if there are no change-points but a time-varying parameter.

This paper proposes a novel approach to modeling change-points within the dynamic linear model framework.
The problem of modeling the change in a parameter value is similar to that of estimating sparse parameters, when the number of parameters is large relative to the number of data points.
That problem is relatively well studied, and there is a growing literature on Bayesian shrinkage priors.
This paper adopts one of this distributions, the Horseshoe Prior (Carvallho) et al. 

\section{Time Varying Parameter Models}
\label{sec:time-vary-param}

A dynamic linear model (DLM) is defined by the following set of equations \parencites{WestHarrison1997}{CommandeurKoopman2007},
\begin{equation}
  \label{eq:9}
  \begin{aligned}[t]
    y_{t} &= F \theta_{t} + \nu_{t} \\
    \theta_{t} &= G \theta_{t - 1} + \omega_{t} 
  \end{aligned}
\end{equation}
where $E(\nu_{t}) = 0$, $E(\omega_{t}) = 0$, and an initial prior distribution for $\theta_{0}$.
This represents a general class of models that includes linear regression (with constant or time-varying parameters) as well as standard ARIMA time-series models.
There are also computationally efficient methods to maximize (Kalman Filter) and sample from (Forward-Filter Backward Sample) DLMs if $\nu$ and $\omega$ are distributed normal.

For the simplicity, in this section, I only consider a special case of the DLM with a single parameter with a time-varying level. 
However, an advantage of this method, and working with state-space models in general, is that these methods can be extend to multiple variables, and varying slopes, etc.
A local-level model is a simple model describing parameter change over time,
\begin{equation}
  \label{eq:8}
  \begin{aligned}
    y_t &= \theta_t + \nu_t \\
    \theta_t &= \theta_{t-1} + \omega_t
  \end{aligned}
\end{equation}
where $v_{t} \sim N(0, \sigma^{2})$ and $E(\omega_t) = 0$.

From \eqref{eq:8}, it is clear that the change in $\theta$ is determined by the distribution of $\omega_{t}$.
In a constant model, $\theta_{t} = \bar{\theta}$ for all $t$. 
This can be represented in \eqref{eq:8} by setting $\omega_{t} = \delta_{0}$, where $\delta_{0}$ is the degenerate distribution at zero.

In the normal (Gaussian) linear dynamic model, $\omega_{t}$ are distributed i.i.d. normal,
\begin{equation}
  \label{eq:2}
  \omega_{t} \sim N(0, \tau^{2})
\end{equation}
Since the normal distribution does not have thick tails, it smooths the evolution of $\theta$ over time.
The use of the normal distribution shrinks the values of $\omega_{t}$, making large jumps unlikely, 
and also means that $\omega \neq 0$ with certainty.

A ``structural break'' or ``change-point'' formulation can be incorporated into a dynamic linear model if $\omega_{t}$ is modeled as a spike-and-slab mixture distribution \parencite{GiordaniKohn2008},
\begin{equation}
  \label{eq:1}
  \omega_{t} \sim p g(\omega_{i}) + (1 - p) \delta_{0} \text{,}
\end{equation}
where $p$ is the prior probability that $\omega_{t} \neq 0$, i.e. there is a regime shift, and 
$g(\omega_{t})$ is a distribution, usually non-information, over the change of the change in $\omega_{t}$ if there is a structural break.
\footnote{If $p$ is constant over time, this replicates the geometric distribution in \textcite{Chib1998}.}

The problem of estimating innovations $\omega_{t}$ is similar to problems of variable selection and shrinkage.
The smoothing approaching of normal DLMs is an example of a shrinkage approach to estimating innovations.
The structural break approach is 

There are several reasons why structural break methods are used.
The first is that the underlying data generating processes in political processes is often marked by large shifts.
The second is that stuctural breaks are easy to intrepret substantively.
The output of the methods provides a small number of breaks which can be interpreted with respect to the events occuring contemporaneously.

A structural break model is a model which imposes sparsity on $\omega_{t}$. 
In other words, for most $t$, $\omega_{t} = 0$, while for a small number of $t$, $\omega_{t} \neq 0$.

A alternative approach is to model $\omega$ with a scale mixture of normal distributions.
\begin{equation}
  \label{eq:6}
  \begin{aligned}[t]
    \omega_{t} | \tau^{2}, \lambda^{2} & \sim N(0, \tau^{2} \lambda_{t}^{2}) \\
    \lambda_{t}^{2} & \sim p(\lambda^{2}_{t})
  \end{aligned}
\end{equation}
where $\tau^{2}$ is a global shrinkage parameter, and $\lambda_{t}^{2}$ are global shrinkage parameters.
Many Bayesian shrinkage priors can be expressed in this form. 

The class of scale-normal mixtures includes many of the suggested Bayesian shrinkage priors.

The $t$-distribution has suggested for dynamic linear model estimation that is robust to structural breaks.

This paper will focus on one particular shrinkage prior, the Horseshoe Prior distribution.

The Horseshoe prior distribution has several characteristics which make it particularly appealing for this problem.

This distribution has several characteristics that are useful.

\begin{equation}
  \label{eq:10}
  E(\omega_{t} | y_{t}, \theta_{t-1}, \lambda_{t}, \tau, \sigma^{2}) =
  \left(
    1 - \frac{1}{1 + \lambda_{t}^{2} \tau^{2}}
  \right) (y_{t} - \theta_{t - 1})
\end{equation}
For large $\psi^{2}$, the posterior mean $\hat\omega_{t}$ under the mixture innovation model is 
approximately $p_{t} (y_{t} - \theta_{t-1})$, where $p_{t}$ is the posterior probability that $\omega_{t} \neq 0$.
In the horseshoe estimator the quantity $\hat \omega_{t} = (1 - \hat \kappa) (y_{i} - \theta_{t})$.
The shrinkage weight behaves similarly to the posterior structural break probability $p_{t}$.
Because of this similarity, Carvalho proposes a simple decision rule under a symmetric 0-1 loss function
to identify signals, which in this case will be structural breaks,
\begin{equation}
  \label{eq:11}
  \text{Reject $H_{0t}$ if $p_{t} = 1 - E(\kappa_{t} | y_{t}, \theta_{t-1}, \tau, \omega^{2})$} > \frac{1}{2}
\end{equation}

\section{Examples}
\label{sec:examples}

\subsection{Nile Flow Data}
\label{sec:nile}

The first example of change-point detection is a classic datset in the state-space and change point literature, the Nile river flow data \textcite{Cobb1978}{Balke1993}{DurbinKoopman2012}
The data consist of annual observations of the flow of the Nile river at Ashwan between 1871 and 1970. 
It is well known that there was a level shift in 1899, both due to the construction of a damn at Ashwan and weather changes.

\subsection{Greenbacks}
\label{sec:greenbacks-graybacks}

\subsection{Election}
\label{sec:election}

Consider the ``Bush'' problem, estimating structural breaks in
presidential approval data. This is the example used in
\textcite{RatkovicEng2010}.


\printbibliography{}

\end{document}

%%% Local Variables: 
%%% mode: latex
%%% TeX-master: t
%%% End: 

%  LocalWords:  Carvallho
